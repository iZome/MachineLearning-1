\documentclass[a4paper,10pt]{article}

\usepackage[a4paper, total={6in, 9in}]{geometry}
\usepackage{graphicx}
\usepackage{svg}
\usepackage{mathtools}

\newlength\Colsep
\setlength\Colsep{10pt}

\usepackage{subfig}
\usepackage{amssymb}
\usepackage{amsfonts}
\usepackage{float}
\usepackage{amsmath}
\usepackage{caption}
\usepackage{subfig}
\usepackage{subfloat}


\usepackage[style=authoryear-comp, backend=biber]{biblatex}
\usepackage{fontspec}

\newcommand{\R}{\mathbb{R}}
\newcommand{\me}{\mathrm{e}}

\graphicspath{{./figures/}}

\begin{document}
\setlength\parindent{0pt}


\title{Perculation Theory}
\clearpage\maketitle
\thispagestyle{empty}

\begin{abstract}
\noindent This project is dedicated to the study of perculation theory: the study of the formation of clusters in a graph.
By taking advantage of that some variables scales as a power law near the phase transition, some of the critical exponents
of perculation theory are estimated and it is found that the form of the graph does not depend on the form but the dimension.
\end{abstract}

\newpage
\clearpage
\setcounter{page}{1}

\section{Introduction}

\begin{figure}[H]
  \subfloat[][Beta]{
    \def\svgwidth{0.5\linewidth}
    {\input{figures/beta.ps_tex}}
  }
  \subfloat[][Alpha]{
    \def\svgwidth{0.5\linewidth}
    {\input{figures/alpha.ps_tex}}
  }
\end{figure}



\begin{figure}[H]
  \centering
  \def\svgwidth{\linewidth}
  {\input{figures/L_q.ps_tex}}
  \caption{A square lattice having red sites and green bonds. This is a lattice with $N = 64$ sites and $M = 128$ bonds.}
\end{figure}

\begin{figure}[H]
  \centering
  \def\svgwidth{\linewidth}
  {\input{figures/task2i.ps_tex}}
  \caption{A square lattice having red sites and green bonds. This is a lattice with $N = 64$ sites and $M = 128$ bonds.}
\end{figure}


\end{document}
