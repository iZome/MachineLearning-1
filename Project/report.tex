\documentclass[a4paper,10pt]{article}

\usepackage[a4paper, total={6in, 9in}]{geometry}
\usepackage{graphicx}
\usepackage{svg}
\usepackage{mathtools}

\newlength\Colsep
\setlength\Colsep{10pt}

\usepackage{subfig}
\usepackage{amssymb}
\usepackage{amsfonts}
\usepackage{float}
\usepackage{amsmath}
\usepackage{caption}
\usepackage{subfig}
\usepackage{subfloat}
\usepackage{verbatim}
\usepackage{amsmath}


\usepackage[style=authoryear-comp, backend=biber]{biblatex}
\usepackage{fontspec}

\newcommand{\R}{\mathbb{R}}
\newcommand{\me}{\mathrm{e}}
\DeclareMathOperator{\EX}{\mathbb{E}}

\graphicspath{{./figures/}}

\begin{document}
\setlength\parindent{0pt}


\title{Assignment 2 - KLVTHO001}
\clearpage\maketitle
\thispagestyle{empty}


\newpage
\clearpage
\setcounter{page}{1}

\section{Introduction}
Many differnet learning algorithms are used in Statistical Sciences in order
to classify hand-written digits. This project is focusing on a number of these
learning approaches, namely Support Vector Machine, Neural Nework and
Tree Based Methods. The dataset provided for this classification problem contains
a response variable which indicated whether a digit is even or odd
and 784 potential predictor variables which represents various digit properties.
All the aforementioned approches are fit to the dataset and the best
predictive algorithm is decided.

\section{Theory}
\subsection{Tree Based Methods}
Bagging, Boosting and Random Forests are all ensemble methods and meta learners. The key
difference between Bagging and Boosting lies in how the two approches use the training set. Bagging
is simply just bootstrap aggregation, which is all about choosing a random sample with replacement,
train the algorithm on each sample seperately and average the predictions in the end. Furthermore,
the key difference between Bagging and Random Forest is that Random Forest has the ability to
improve variance by reducing the correlation between each tree in the forest. This is accomplished
by randomly selecting a feature-subset for each split at each node. This is the main reason why
Random Forests generally will generalize better as the number of trees grows. Furthermore,
The key difference between these methods is that a Random Forest and Bagging is trained
in parallel, i.e. each model is build independantly. In contrast,
Boosting builts the models in a sequential way, making each model dependant on
the previous ones. \\

\subsection{Support Vector Machine }
Support Vector Machine is a supervised binary classification algorithm. It
attempts to find a hyperplane in a high dimensional space that can seperate
the two classes of data by the largest margin. In order to achieve this, the
Support Vector Machine will use a kernel to find the hyperplane that separates
the data best. \\

\subsection{Artificial Neural Network}
A neural network is also a supervised classifier. The network consists
of several components interconnected and organized in layers. Each layer
consists of neurons, which itself is a simple classifier. The input data
is fed to the network and will pass through in a forward-feed manner.
Furthermore, the training part is often done with the \textit{Back Propagation}
algorithm which helps find the optimal weights of each neuron in
the layers of the network.

\section{Method}
\subsection{Tree Based Methods}
The \texttt{randomForest} and \texttt{gbm}-packages is used for training
the Tree Based Methods. Furthermore,
the models trained can be viewed in Table {\ref{table:trained_trees}}\\

\begin{table}[H]
\centering
\begin{tabular}{l*{7}{c}r}
  Method              & Number of trees & Mtry & Interaction Depth & Learning Rate & Bag Fraction & CV Folds\\
  \hline
  Random Forest & 500 \\
  Bagging & 500 & 32 \\
  Boosting & 50.000 &  & 2 & 0.001 & 1 & 10 \\
  Regression Tree & 1 &  &   &   &  & 10  \\
\end{tabular}
\caption{Trained Models. If a field is empty, the default package settings are used.}
\label{table:spam_models}
\end{table}

\subsection{Support Vector Machine}
A radial based kernel is used for the Support Vector Machine. In addition,
Principal Component Analysis is performed in order to reduce the
dimensionality of the problem. Furthermore, the \texttt{caret}-package in
\texttt{R} is used to fit a model to the dataset. For the Support Vector Machine
a number of different soft constraints where tested,
thereof~$C$~$\in\ \{0, 0.5, 1, 1.5, 2, 3\}$ \\

\subsection{Artificial Neural Network}
For the Neural Network, the \texttt{h2o}-package is used in order to
classify the digits. A number of different hyperparameters are tested
using a grid and the best trained model is then used for further
analysis. The trained models for the  Neural Network the models can
be seen Table {\ref{table:trained_models_nn}}.

\begin{table}[H]
\centering
\begin{tabular}{l*{3}{c}r}
  Parameters & Value \\
  \hline
  Epochs & 5, 10\\
  Hidden & [512, 128], [218,42]\\
  Rate & 0.005, 0.01\\
  Input Dropout Ratio & 0.1\\
  Nfolds & 10\\
  Stopping Rounds & 3\\
  Stopping Metric & Misclassification\\
  Stopping Tolerance & 0.02\\
\end{tabular}
\caption{Trained Models}
\label{table:trained_models_nn}
\end{table}

Where $Ephocs$  is the number of times to iterate (stream) the dataset,
$Hidden$ is the hidden layer sizes, $Rate$ is the
the learning rate, $Input\ Droput\ Ratio$ is
specifying the input layer dropout ratio to improve generalization,
$Nfolds$ is the the number of folds for cross-validation. Furthermore,
the network stops training when misclassification rate,
$Stopping\ Metric$, does not improve for the specified number of
training rounds, based on a simple moving average. Lastly,
the $Stopping\ Tolerance$
specifies the relative tolerance for the metric-based
stopping to stop training if the improvement is less than this value. \\






\section{Results}
\begin{figure}[H]
  \subfloat[][]{
  \def\svgwidth{0.5\linewidth}
  {\input{figures/cumsum.ps_tex}}
  \label{fig:cumsum}}
  \hfill
  \subfloat{\def\svgwidth{0.5\linewidth}
  {\input{figures/pc.ps_tex}}
  \label{fig:pc1_pc3}} \\
  \subfloat{ \def\svgwidth{0.5\linewidth}
  {\input{figures/grid_acc.ps_tex}}
  \label{fig:grid_acc}}
  \hfill
  \subfloat{\def\svgwidth{0.5\linewidth}
  {\input{figures/e_out.ps_tex}}
  \label{fig:num_componentes}}
\end{figure}

\section{Results}
\begin{figure}[H]
  \subfloat[][]{
  \def\svgwidth{0.5\linewidth}
  {\input{figures/bag_rf_cv.ps_tex}}
  \label{fig:cumsum}}
  \hfill
  \subfloat[][]{
  \def\svgwidth{0.5\linewidth}
  {\input{figures/mse_grid.ps_tex}}
  \label{fig:cumsum}}
  \hfill
\end{figure}

\begin{table}[h]
\centering
\begin{tabular}{l*{4}{c}r}
  Class & Even & Odd & Total & Error Rate\\
  \hline
  Even & 222 & 17 & 239 & 0.034 \\
  Odd & 20 & 241 & 261 & 0.04\\
  Total & 242 & 258 & 500 & 0.074\\
\end{tabular}
\caption{Random Forest confusion matrix}
\label{table:rf_conf}
\end{table}

\begin{table}[h]
\centering
\begin{tabular}{l*{4}{c}r}
  Class              & Even & Odd & Total & Error Rate\\
  \hline
  Even & 214 & 32 & 246 & 0.064 \\
  Odd & 28 & 226 & 254 & 0.056\\
  Total & 242 & 258 & 500 & 0.12\\
\end{tabular}
\caption{Bagging confusion matrix}
\label{table:bag_conf}
\end{table}



\section{Discussion}


\section{Appendix}
\begin{verbatim}


\end{verbatim}


\end{document}
